\documentclass{article}
\usepackage[utf8]{inputenc}
\usepackage[spanish]{babel}
\usepackage{listings}
\usepackage{graphicx}
\graphicspath{ {images/} }
\usepackage{cite}

\begin{document}

\begin{titlepage}
    \begin{center}
        \vspace*{1cm}
            
        \Huge
        \textbf{Taller de Informatica II}
            
        \vspace{0.5cm}
        \LARGE
        Calistenia
            
        \vspace{1.5cm}
            
        \textbf{Jesús Antonio Ibarra Agudelo}
            
        \vfill
            
        \vspace{0.8cm}
            
        \Large
        Despartamento de Ingeniería Electrónica y Telecomunicaciones\\
        Universidad de Antioquia\\
        Medellín\\
        Marzo de 2021
            
    \end{center}
\end{titlepage}

\tableofcontents
\newpage
\section{Introducción}\label{intro}
un computador es una herramienta que está en capacidad de capturar información, almacenarla y procesarla, por tanto, programar un computador es decir que hay una información en cierto lugar almacenado para procesarlo de cierta forma deseada y obtener resultados.

Con este ejercicio veremos que lo más importante a la hora de programar un computador son las instrucciones que le damos. Tenemos que tomarnos un tiempo para pensar y analizar bien el problema para dar una serie de instrucciones correctas en el momento correcto. Estas instrucciones describirán pasos consistentes y deberán ser lo más específico posible para poder programar bien. Necesitamos pensar en qué hacer y establecer un orden de instrucciones, pero en esto fallamos, porque no nos tomamos el tiempo suficiente para pensar en el problema, una vez hallamos arreglado esta falla, podremos programar un computador de manera exitosa.

\newpage

\section{Contenido} \label{contenido}
A continuacion, se mostraran una serie de pasos que se utilizo para realizar este trabajo. Se trato de que estas instrucciones fueran lo mas especificas posible aunque debo admitir que pueden y se mejoraran para proximas actividades.

\subsection{Instrucciones}
A partir de un estado inicial (ambas tarjetas debajo de un papel) hasta el estado final (una piramide con las tarjetas sobre el papel) se emplearon las siguientes instrucciones:

\begin{enumerate}
    \item Del estado inicial, coger la hoja y moverla a un lugar sobre la mesa de tal forma de que no quede encima de las tarjetas.
    \item Coger las tarjetas y colocarlas verticalmente sobre la hoja.
    \item Despues, ponga el pulgar en el costado lateral de las tarjetas y separelas, formando un triangulo sin soltar las tarjetas.
    \item Teniendo en cuenta el sentido de izquierda y derecha, desplace la tarjeta derecha aun mas hacia la derecha permaneciendo la inclinacion hacia la izquierda, mientras que la tarjeta izquierda sigue inclinandose hacia la derecha, formando un trianfulo mas ancho.
    \item Sin soltar las tarjetas, siga desplazando la tarjeta derecha hasta que el lado superior de la tarjeta izquierda este por debajo uno o dos centimetros del lado superior de la tarjeta derecha, mientras aun permanece la forma trianfular.
    \item Retire la mano para que quede en el estado final.
\end{enumerate}
\newpage
\section{Conclusion} \label{conclusion}
Para concluir, podemos ver que tan importante son las instrucciones especificas y correctas a la hora de progrmar, ya que nos ayuda a que el proceso  de progrmar un computador sea mucho mas facil y eficiente, por tanto, hay que seguir recalcando la importancia de tomarse un tiempo para analizar el problema y buscar la mejor solucion, para que al final, las instrucciones sean las que se deban aplicar y el programa resultante sea todo un exito.

\end{document}
